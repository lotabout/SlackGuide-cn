
\chapter{操作文件及目录}
\label{chap:handlingFilesAndDirectories}

\begin{flushleft}
\rule[0mm]{\textwidth}{.1pt}
\end{flushleft}

Linux的目标是尽可能地类Unix。传统上,Unix操作系统是面向命令行的。
Slackware中的确有图形的用户接口,但命令行仍是控制系统的主要方法。因此,
理解一些基本文件管理命令是很重要的。

接下来的几节中,我们会解释常见的文件管理命令并举一些使用它们的例子。还
有许多其它的命令,但我们介绍的命令能让你逐渐上手。另外,我们只对这些命
令进行简要介绍,要了解这些命令的细节,请查阅命令附带的man手册。


\section{导航命令:ls、cd及pwd}
\label{sec:handlingFilesAndDirectories:navigation}

\subsection{ls}
\label{sec:handlingFilesAndDirectories:navigation:ls}
这个命令是``list''的缩写,它的作用是列出一个文件夹中的文件。如果你熟悉
Windows及DOS,那么ls就像是其中的\texttt{dir}命令。如果不带任何参数,
\texttt{ls\textbf{(1)}}会列出当前目录下的文件。要查看根目录下的文件,
可以使用如下命令:
\begin{Verbatim}[frame=single,commandchars=\\\{\}]
darkstar~\% \textbf{cd /}
darkstar~\% \textbf{ls}
bin   dev  home  lost+found  mnt  proc  sbin  sys  usr
boot  etc  lib   media       opt  root  srv   tmp  var
\end{Verbatim}

对于这个输出,多数人的问题是没办法轻易区分哪个是目录,哪个是文件。一些
用户喜欢让\texttt{ls}在每个列表项的末尾加上一个类型识别符,如:
\begin{Verbatim}[frame=single,commandchars=\\\{\}]
darkstar~\% \textbf{cd /}
darkstar~\% \textbf{ls -FC}
bin/   dev/  home/  lost/+found/  mnt/  proc/  sbin/  sys/  usr/
boot/  etc/  lib/   media/       opt/  root/  srv/   tmp/  var/
\end{Verbatim}
目录名之后会加上一个反斜杠(`/'),可执行文件名之后会加上一个星号(`*')
等等。

\texttt{ls}也可以用来显示其它的文件数据信息。例如,要显示一个文件的创
建日期、所有者及权限等,可以使用\texttt{ls}显示长列表:
\begin{Verbatim}[frame=single,commandchars=\\\{\}]
darkstar~\%\texttt{ls -l}
# \textbf{ls -l}
total 88
drwxr-xr-x   2 root root  4096 Apr 30  2007 bin/
drwxr-xr-x   2 root root  4096 Jun 19 13:56 boot/
drwxr-xr-x  15 root root  5180 Jun 21 21:37 dev/
drwxr-xr-x  78 root root 12288 Jun 21 12:03 etc/
drwxr-xr-x   4 root root  4096 Feb 28  2011 home/
drwxr-xr-x   6 root root  4096 Mar 19  2011 lib/
drwx------   2 root root 16384 Jun 15 21:35 lost+found/
drwxr-xr-x  16 root root  4096 Jun 21 12:04 media/
drwxr-xr-x  10 root root  4096 Sep 26  2006 mnt/
drwxr-xr-x   3 root root  4096 Jun 20 08:25 opt/
dr-xr-xr-x 110 root root     0 Jun 21 20:03 proc/
drwx--x---  11 root root  4096 Jun 23 15:28 root/
drwxr-xr-x   2 root root 12288 Jun 20 08:29 sbin/
drwxr-xr-x   2 root root  4096 Jun 15 22:12 srv/
drwxr-xr-x  13 root root     0 Jun 21 20:03 sys/
drwxrwxrwt   4 root root  4096 Jun 21 09:40 tmp/
drwxr-xr-x  15 root root  4096 Apr  5  2011 usr/
drwxr-xr-x  17 root root  4096 Mar 19  2011 var/
\end{Verbatim}

假设你想得到一个包含隐藏文件的当前目录的列表,可以使用如下命令:
\begin{Verbatim}[frame=single,commandchars=\\\{\}]
darkstar~\% \textbf{ls -a}
./   bin/   dev/  home/  lost+found/  mnt/  proc/  sbin/  sys/  usr/
../  boot/  etc/  lib/   media/       opt/  root/  srv/   tmp/  var/
\end{Verbatim}
以点号开头的文件(也称为点文件)在运行\texttt{ls}命令时会隐藏,只有在
以\texttt{-a}为参数运行ls才可以看到。

在ls的man手册上可以看到它还有许许多多的选项。运行ls的时候别忘了还可以
加上选项。


\subsection{cd}
\label{sec:handlingFilesAndDirectories:navigation:cd}

\texttt{cd}命令的作用是改变当前的工作目录。只要输入\texttt{cd}命令,以
及想要切换到的目录即可。下面是一些例子:
\begin{Verbatim}[frame=single,commandchars=\\\{\}]
darkstar:~\$ \textbf{cd /bin}
darkstar:/bin\$ \textbf{cd usr}
bash: cd: usr: No such file or directory
darkstar:/bin\$ \textbf{cd /usr}
darkstar:/usr\$ \textbf{ls}
bin
darkstar:/usr\$ \textbf{cd bin}
darkstar:/usr/bin\$
\end{Verbatim}

注意到,如果没有前置的斜杠(`/'),则切换目录时是相对于当前目录进行的。
另外,如果运行\texttt{cd}命令是不带任何参数,则会切换到用户的主目录中。

\texttt{cd}命令与其它命令不同,因为它是一个shell内置的命令。这个特点现
在对你可能没什么影响,一般而言,这意味着这个命令没有man手册。相反的,
你可以使用shell的帮助,如下:
\begin{Verbatim}[frame=single,commandchars=\\\{\}]
darkstar~\% \textbf{help cd}
\end{Verbatim}
它会打印出\textbf{cd}命令可用的选项。

\subsection{pwd}
\label{sec:handlingFilesAndDirectories:navigation:cd}
\texttt{pwd}的全称是打印工作目录(print working directory)。它的作用
就是显示当前位置。使用\texttt{pwd}命令只要输入\textbf{pwd}即可。例子如
下:
\begin{Verbatim}[frame=single,commandchars=\\\{\}]
darkstar~\% \textbf{cd /bin}
darkstar~\% \textbf{pwd}
/bin
darkstar~\% \textbf{cd /usr}
darkstar~\% \textbf{cd bin}
darkstar~\% \textbf{pwd}
/usr/bin
\end{Verbatim}

\section{分页程序:more、less及most}
\label{sec:handlingFilesAndDirectories:pagers}

\subsection{more}
\label{sec:handlingFilesAndDirectories:pagers:more}
\texttt{more\textbf{(1)}}命令被称为分页工具。有时,一个命令的输出太大
了,一个屏幕不足以显示。而单独的命令并不知道如何控制输出使它适合在一个
屏幕上显示。它们将这个任务留给了分页工具。

\texttt{more}命令将其它命令的输出分成几个单独的屏幕,一次输出一个屏幕,
直到用户按下空格键才输出下一屏,按回车键则向前显示一行。下面是个很好的
例子:
\begin{Verbatim}[frame=single,commandchars=\\\{\}]
darkstar~\% \texttt{cd /usr/bin}
darkstar~\% \texttt{ls -l}
\end{Verbatim}

这个命令的输出会在屏幕上滚动一会。要将该输出一屏一屏的输出,只要用管道
将它传递给\texttt{more}命令即可:
\begin{Verbatim}[frame=single,commandchars=\\\{\}]
darkstar~\% \textbf{ls -l | more}
\end{Verbatim}
中间的符号为管道符(`|'),上述管道的含义是将\texttt{ls}的输出作为
\texttt{more}的输入。除了\texttt{ls},还可以用管道将其它几乎所有的东西
传递给\texttt{more}命令。管道在第
\ref{sec:shell:commandLine:redirectionAndPiping}节中有介绍。

\subsection{less}
\label{sec:handlingFilesAndDirectories:pagers:less}
\texttt{more}命令很方便,只是有个致命的缺陷,就是如果翻到下一页就再也
不能回到上一页。试想如果你在查看一个很大的文件,不小心错过了某一行,那
就不得不重头查找,急煞人也。好在人们开发了\texttt{less\textbf{(1)}}命
令,来解决这个问题。它的工作方式与\texttt{more}一样,所以上面的例子在
这里也适用。所以\texttt{less}的功能远不止\texttt{more}。Joost Kremers
是这样说的:
\begin{quote}
  \texttt{less} is more, but more \texttt{more} than \texttt{more} is,
  so \texttt{more} is less \texttt{less}, so use more \texttt{less} if
  you want less \texttt{more}.

  \texttt{less}与\texttt{more}类似,但比\texttt{more}的功能更强大,所
  以\texttt{more}的功能反而更少,因此,如果你想更少使用\texttt{more},
  那么请更多地使用\texttt{less}。
\end{quote}

\subsection{most}
\label{sec:handlingFilesAndDirectories:pagers:most}
\texttt{more}和\texttt{less}已经停止开发了,而
\texttt{most\textbf{(1)}}继承了它们的衣钵。如果说\texttt{less}是
\texttt{more}的加强版,那么\texttt{most}就是\texttt{less}的加强版。其
它的分页程序一次只能查看一个文件,而\texttt{most}能同时查看任意数量的
文件,前提是每个文件至少两行以上。\texttt{most}提供了很多选项,同样,
也请自己查阅man手册。

\section{简单输出:cat及echo}
\label{sec:handlingFilesAndDirectories:simpleOutput}

\subsection{cat}
\label{sec:handlingFilesAndDirectories:simpleOutput:cat}
\texttt{cat}(1) 是``concatenate(连接)''的缩写,它最初的设计目的就是
将多个文件合并成一个文件,但它同时被用于其它的许多目的。

要想将两个或更多的文件合并成一个,只要将想合并文件的文件名传递给
\texttt{cat}命令,并将输出重定向到另一个文件。\texttt{cat}还能与标准输
入标准输出协同工作,所以我们需要使用shell的重定向符。照例举个例子:
\begin{Verbatim}[frame=single, commandchars=\\\{\}]
\% \textbf{cat file1 file2 file3 > bigfile}
\end{Verbatim}
这个命令将\texttt{file1}、\texttt{file2}及\texttt{file3}的内容合并在一
起,最后的输出传递给标准输出,由于使用了重定向,标准输出的内容最终存储
到\texttt{bigfile}文件中。

我们可以使用\texttt{cat}命令来显示文件的内容。许多人都使用\texttt{cat}
将文件输出到\texttt{more}或\texttt{less}命令中进行查看,如:
\begin{Verbatim}[frame=single, commandchars=\\\{\}]
\% \textbf{cat file1 | more}
\end{Verbatim}
这个命令会显示\texttt{file1}文件的内容,并通过管道将它传递给
\texttt{more}命令,因此我们就可以一屏一屏地查看该文件。

\texttt{cat}的另一个用法是用来复制文件。例如:
\begin{Verbatim}[frame=single, commandchars=\\\{\}]
\% \textbf{cat /bin/bash > ~/mybash}
\end{Verbatim}
执行该命令后,\path{/bin/bash}程序就会被复制到我们的主目录下,并命名为
\texttt{mybash}。

\texttt{cat}还有许多用法,这里介绍的只是很小的一部分,由于\texttt{cat}
利用了标准输入及标准输出,因此,它适用于shell脚本中,作为其它复杂命令
的一部分。


\subsection{echo}
\label{sec:handlingFilesAndDirectories:simpleOutput:echo}
\texttt{echo}(1)命令的作用是在屏幕上显示特定的文本。我们可以在
\texttt{echo}命令之后指定要显示的字符串。默认情况下,\texttt{echo}会显
示我们给出的字符串并在之后打印一个换行符。当然,我们也可以为其传递参数
\texttt{-n}来告诉\texttt{echo}不打印末尾的新行。\texttt{-e}选项则告诉
\texttt{echo}解释字符串中的转义符。

\section{新建命令:touch及mkdir}
\label{sec:handlingFilesAndDirectories:creation}

\subsection{touch}
\label{sec:handlingFilesAndDirectories:creation:touch}
\texttt{touch}(1)命令的初衷是修改一个文件的时间戳,使用\texttt{touch}
命令,我们就可以修改一个文件的访问时间戳及修改时间戳。而如果一个文件之
前并不存在,则\texttt{touch}会以给定的文件名创建一个空的文件。要将一个
文件标记为当前的系统时间,只要使用如下命令即可:
\begin{Verbatim}[frame=single, commandchars=\\\{\}]
\% \textbf{ls -al file1}
-rw-r--r-- 1 root root 0 Jun 25 10:42 file1
\% \textbf{touch file1}
\% \textbf{ls -al file1}
-rw-r--r-- 1 root root 0 Jun 25 10:43 file1
\end{Verbatim}

\texttt{touch}命令有很多参数,包括选择修改哪个时间戳、修改成什么时间及
一些其它的选项。man手册中有详细的介绍。

\subsection{mkdir}
\label{sec:handlingFilesAndDirectories:creation:mkdir}
\texttt{mkdir}(1)的作用是创建一个新的目录,使用方法是在\texttt{mkdir}
后加上想要创建的目录名。下面的例子将会在当前目录下创建一个新的名为
\texttt{hejaz}的目录:
\begin{Verbatim}[frame=single, commandchars=\\\{\}]
\% \textbf{mkdir hejaz}
\end{Verbatim}

当然,我们也可以指定一个路径,如:
\begin{Verbatim}[frame=single, commandchars=\\\{\}]
\% \textbf{mkdir /usr/local/hejaz}
\end{Verbatim}

\texttt{-p}选项会告诉\texttt{mkdir}创建目录时一同创建它的父目录。例如,
上面的例子中,如果\path{/usr/local}目录不存在,则这条命令就会执行失败,
而加上\texttt{-p}选项,则意味着先创建\path{/usr/local},之后再创建
\path{/usr/local/hejaz}目录。
\begin{Verbatim}[frame=single, commandchars=\\\{\}]
\% \textbf{mkdir -p /usr/local/hejaz}
\end{Verbatim}

\section{复制与移动}
\label{sec:handlingFilesAndDirectories:copyAndMove}

\subsection{cp}
\label{sec:handlingFilesAndDirectories:copyAndMove:cp}
\texttt{cp}命令(copy的缩写)的作用就是复制文件。DOS用户会发现它与DOS
的\texttt{copy}命令类似。\texttt{cp}命令的选项很多,所以在使用之前还是
建议查看它的man手册。这里会介绍一些最常用的选项。

首先,\texttt{cp}最常用的方法就是将一个文件复制到另一个位置。如下例:
\begin{Verbatim}[frame=single, commandchars=\\\{\}]
\% \textbf{cp hejaz /tmp}
\end{Verbatim}
这个命令就是将当前目录下的\path{hejaz}文件复制到\path{/tmp}目录下。

许多用户在复制时还希望保留文件的时间戳,如下面的例子:
\begin{Verbatim}[frame=single, commandchars=\\\{\}]
\% \textbf{cp -a hejaz /tmp}
\end{Verbatim}
这个命令就保证了复本的时间戳与原文件一致。

如果想复制整个目录,则使用下面的命令:
\begin{Verbatim}[frame=single, commandchars=\\\{\}]
\% \textbf{cp -R mydir /tmp}
\end{Verbatim}
该命令的效果是将\path{mydir}目录复制到\path{/tmp}目录下。

另外,像前面的例子一样,如果在复制文件或目录的过程中,同时希望保持原文
件的时间戳,那么使用\texttt{cp -p}命令。
\begin{Verbatim}[frame=single, commandchars=\\\{\}]
\% \textbf{ls -l file}
-rw-r--r-- 1 root vlad 4 Jan 1 15:27 file
\% \textbf{cp -p file /tmp}
\% \textbf{ls -l /tmp/file}
-rw-r--r-- 1 root vlad 4 Jan 1 15:27 file
\end{Verbatim}

\texttt{cp}还有许多其它的选项,请自行参阅man手册。

\subsection{mv}
\label{sec:handlingFilesAndDirectories:copyAndMove:mv}
\texttt{mv}命令的作用是将文件从一个位置移动到另一个位置,听起来是不是
很简单?
\begin{Verbatim}[frame=single, commandchars=\\\{\}]
\% \textbf{mv oldfile /tmp/newfile}
\end{Verbatim}

\texttt{mv}有一些很有用的参数,在man手册中有详细的描述,但实际使用中一
般几乎不怎么用到。

注意,在Linux,并没有重命名的命令,因此,所有的重命令都是通过
\texttt{mv}命令实现的。如下面的例子:
\begin{Verbatim}[frame=single, commandchars=\\\{\}]
\% \textbf{mv oldfile newfile}
\end{Verbatim}
该命令的作用就是将当前目录下名为\path{oldfile}的文件重命名为\path{newfile}。

\section{删除命令:rm及rmdir}
\label{sec:handlingFilesAndDirectories:deletion}

\subsection{rm}
\label{sec:handlingFilesAndDirectories:deletion:rm}

\texttt{rm}(1)的作用是删除文件及目录树。DOS用户会发现它的功能与DOS下的
\texttt{del}及\texttt{deltree}命令类似。如果不小心使用,\texttt{rm}可
能造成意想不到的后果。尽管有一些方法能还原最近删除的文件,但是方法很复
杂(而且代价昂贵),并且不在本书的范围内。

删除一个单独的文件,只要将文件名传递给\texttt{rm}即可:
\begin{Verbatim}[frame=single, commandchars=\\\{\}]
\% \textbf{rm file1}
\end{Verbatim}

如果一个文件没有可写权限,则执行\texttt{rm}时会得到一个提示,说该文件
是写保护,是否强制删除。如果想跳过这个提示信息,在执行\texttt{rm}时加
上\texttt{-f}选项,如:
\begin{Verbatim}[frame=single, commandchars=\\\{\}]
\% \textbf{rm -rf file1}
\end{Verbatim}

如果想删除整个文件夹,可以联合使用\texttt{-r}及\texttt{-f}选项。下面是
一个例子,教你如何删除整个硬盘,\textbf{请不要在自己的机器上尝试}。下
面贴出代码:
\begin{Verbatim}[frame=single, commandchars=\\\{\}]
# \textbf{rm -rf /}
\end{Verbatim}

使用\texttt{rm}时一定要小心,不然很可能走火,它也有一些命令行参数,和
以往一样,在man手册中有详细介绍。

\subsection{rmdir}
\label{sec:handlingFilesAndDirectories:deletion:rmdir}

\texttt{rmdir}(1)的作用是删除目录,但在删除之前要确保该目录是空的。该
命令的语法如下:
\begin{Verbatim}[frame=single, commandchars=\\\{\}]
\% \textbf{rmdir <directory>}
\end{Verbatim}

下面这个例子的作用是删除当前目录下的\path{hejaz}子目录:
\begin{Verbatim}[frame=single, commandchars=\\\{\}]
\% \texttt{rmdir hejaz}
\end{Verbatim}

如果要删除的目录是不存在的,\texttt{rmdir}会提醒你。你也可以手工指定要
删除目录的完整路径,如下目录:
\begin{Verbatim}[frame=single, commandchars=\\\{\}]
\% \textbf{rmdir /tmp/hejaz}
\end{Verbatim}
这个命令的作用是将\path{/tmp}目录下的\path{hejaz}目录删除。

还可以通过传递\texttt{-p}选项,在删除一个目录的同时,删除它的父目录。
\begin{Verbatim}[frame=single, commandchars=\\\{\}]
\% \texttt{rmdir -p /tmp/hejaz}
\end{Verbatim}
这个命令首先删除\path{/tmp}下的\path{hejaz}目录,如果成功删除,则
\texttt{rmdir}会尝试删除\path{/tmp}目录,\texttt{rmdir}会递归地执行此
操作直到遇到错误或成功删除整个目录树。

\section{文件别名:ln}
\label{sec:handlingFilesAndDirectories:aliasing}
\texttt{ln}(1)(link的缩写)的作用是创建文件间的链接。链接分为硬链接或
软链接(又称为符号链接)。两种链接的区别在第
\ref{sec:filesystemStructure:links}节中有介绍。例如我们想创建一个到
\path{/var/media/mp3}的链接,并将其放在主目录下,那么可以执行如下命令:
\begin{Verbatim}[frame=single, commandchars=\\\{\}]
\% \textbf{ln -s /var/media/mp3 ~/mp3}
\end{Verbatim}

\texttt{-s}选项告诉\texttt{ln}要创建符号链接。下一个参数代表链接源的位
置,最后一个参数则代表这个链接的名字是什么。这个例子中,我们在主目录下
创建了一个名为\path{mp3}的文件,指向\path{/var/media/mp3},通过改变第
三个参数,你可以为该链接指定任意的名字。

创建一个硬链接则更为容易,只要去掉\texttt{-s}选项即可。但硬链接不允许
指向目录,或跨越文件系统(即分区)指定,如创建一个名为
\path{/usr/bin/email}的指向\path{/usr/bin/mutt}的链接,可以输入以下命
令:
\begin{Verbatim}[frame=single, commandchars=\\\{\}]
# \textbf{ln /usr/bin/mutt /usr/bin/email}
\end{Verbatim}

%TODO:详细解释链接的原理及区别





%%% Local Variables: 
%%% mode: latex
%%% TeX-master: "../SlackGuide"
%%% End: 
