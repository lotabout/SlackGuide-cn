
\chapter{必要的系统管理}
\label{chap:systemAdministration}

\begin{flushleft}
\rule[0mm]{\textwidth}{.1pt}
\end{flushleft}

嘿!嘿!嘿!嘿!嘿!!我知道你在想什么。``我又不是系统管理员,我连想都
没想过要变成一个系统管理员!''

事实上,只要你有一台机器的\texttt{root}密码,你就是一个系统管理员。也
许你控制的机器也就是一个桌面系统,上面就一两个用户,也或者你在控制着一
个拥有成百上千用户的大型服务器。不管怎么样,我们还是要学习如何管理用户、
如何安全地关闭计算机。这些任务可能看似简单,但脑子里还是要记得一些名言
警句的。

\section{用户及用户组}
\label{sec:systemAdministration:usersAndGroups}
正如在第\ref{chap:shell}章中说到的一样,我们平常工作时不应该以
\texttt{root}用户登陆。而应该创建一个普通用户进行日常的使用,只在系统
执行系统管理任务时才使用\texttt{root}用户。要创建一个新用户,我们可以
使用Slackware提供的工具,也可以手工修改密码文件。

\subsection{工具脚本}
\label{sec:systemAdministration:usersAndGroups:suppliedScripts}
管理用户及用户组最简单的方法是使用系统提供的脚本及程序。Slackware中包
括了\texttt{adduser}、\texttt{userdel}(8)、\texttt{chfn}(1)、
\texttt{chsh}(1)及\texttt{passwd}(1)等工具来处理用户帐户。而提供
\texttt{groupadd}(8)、\texttt{groupdel}(8)及\texttt{groupmod}(8)来处理
用户组。除了\texttt{chfn}、\texttt{chsh}及\texttt{passwd},其它程序一
般情况下只能由\texttt{root}运行。因此它也位于\path{/usr/sbin}目录下。
\texttt{chfn}、\texttt{chsh}及\texttt{passwd}可以由任意用户运行,也因
此位于\path{/usr/bin}目录下。

我们可以通过\texttt{adduser}程序来添加用户。我们会先执行一遍该程序,演
示该程序的所有提问及相应的含义。默认的答案由括号括起,几乎所有的问题都
可以选择默认答案,除非你真的想对它进行修改。

\begin{Verbatim}[frame=single, commandchars=\\\{\}]
# \textbf{adduser}
Login name for new user []: jellyd
\end{Verbatim}
这个问题是填写要添加用户使用的用户名。传统地,用户名是8个字符以下的由
任意小写字符组成的。(当然也可以使用8个以上的字符,还可以使用数字,但
除非别无他法,否则请遵守传统。)

我们也可以在执行脚本时就指定用户名:
\begin{Verbatim}[frame=single, commandchars=\\\{\}]
# \textbf{adduser jellyd}
\end{Verbatim}

不论以上述哪种方式指定用户名,adduser都会提示输入为新用户使用的用户ID:
\begin{Verbatim}[frame=single, commandchars=\\\{\}]
User ID (’UID’) [ defaults to next available ]:
\end{Verbatim}

用户ID是Linux系统内部实际用来识别用户的方法。每个用户都有唯一的ID,
Slackware中是从1000开始的。你可以手动为用户指定ID,也可以选择默认的方
式,默认情况下使用的是下一个可用的ID。

\begin{Verbatim}[frame=single, commandchars=\\\{\}]
Initial group [users]:
\end{Verbatim}

所有的用户默认都加入\texttt{users}组。你也许想将新用户加入到一个不同的
用户组,除非你清楚自己所做的行为,否则我们不建议你这么做。

\begin{Verbatim}[frame=single, commandchars=\\\{\}]
Additional groups (comma separated) []:
\end{Verbatim}

这个提问允许我们将新用户加入到一些额外的组中。用户组的作用是对一个类别
的用户提供某些权限。还记得讲解文件权限时有一个权限组就是拥有文件的组的
权限吗?通过设置某个文件的所有(属)组,并加某个用户加入特定的用户组中,
就可以赋予该用户这个文件的某些权限。一个用户可以同时属于多个用户组。例
如你建立了一些诸如用来修改网站、玩游戏或其它功能的用户组时,这个功能就
很有用了。例如,一些站点定义了\texttt{wheel}组,作为唯一能使用
\texttt{su}命令的给。或者,如Slackware默认安装后,使用\texttt{sys}组来
为用户提供使用内置声卡播放声音的权限。

如果是一个新手,不知道应该加入什么组,那么可以按方向键的``上''键,采用
默认设置的一些额外用户组。

\begin{Verbatim}[frame=single, commandchars=\\\{\}]
Shell [ /bin/bash ]
\end{Verbatim}

\texttt{bash}是Slackware的默认shell,对于绝大多数人而言也是适用的。如
果你是从Unix转过来的,那么可能更熟悉其它类型的shell,这种情况下,你可
以现在就指定要使用的shell,也可以之后通过命令\texttt{chsh}来修改默认
shell。

\begin{Verbatim}[frame=single, commandchars=\\\{\}]
Expiry date (YYYY-MM-DD) []:
\end{Verbatim}

我们可以设置帐号过期的日期。默认情况下是不设置过期时间的。当然,如果你
喜欢,可以自己修改过期的日期。这个选项对于运行ISP的人们可能较为有用,
因为它们可以设置一个帐号在某个时间过期,直到收到下一年的费用才为他们续
期。

\begin{Verbatim}[frame=single, commandchars=\\\{\}]
New account will be created as follows:

---------------------------------------
Login name.......:  jellyd
UID..............:  [ Next available ]
Initial group....:  users
Additional groups:  [ None ]
Home directory...:  /home/jellyd
Shell............:  /bin/bash
Expiry date......:  [ Never ]

This is it... if you want to bail out, hit Control-C.  Otherwise, press
ENTER to go ahead and make the account.
\end{Verbatim}

现在我们会看到刚才输入的所有关于新用户的信息,并且得到一次中途退出的机
会,如果刚才的输入有误,那么可以按Ctrl+C结束并重新来过,否则,我们可以
按回车键,\texttt{adduser}就会开始创建帐号。
\begin{Verbatim}[frame=single, commandchars=\\\{\}]
Creating new account...
Changing the user information for jellyd
Enter the new value, or press return for the default
        Full Name []: Jeremy
        Room Number []: Smith 130
        Work Phone []:
        Home Phone []:
        Other []:
\end{Verbatim}
下面的这些内容都是可选的。如果不想填就可以不填。而用户可以随时使用
\texttt{chfn}命令来改变这些设定。但一般而言,我们会发现至少应该输入用
户的全名和电话号码,防止之后可能要联系该用户。

\begin{Verbatim}[frame=single, commandchars=\\\{\}]
Changing password for jellyd
Enter the new password (minimum of 5, maximum of 127 characters)
Please use a combination of upper and lower case letters and numbers.
New password:
Re-enter new password:
Password changed.
Account setup complete.
\end{Verbatim}

之后我们被提示要为新用户设置一个密码,如果新用户当前不在,我们可以先设
置一个默认的密码,并告诉用户及时修改。

\begin{quote}
  \textbf{注意:}选择一个密码:防止电脑被黑的第一层防线就是选择一个安
  全的密码。不应该选择很容易猜到的密码,因为这样别人就能很容易地闯进你
  的系统。理想的安全密码应该是一个随机的字串,其中包括大写、小写字母、
  数字及一些随机的字符。(根据你所登陆的系统,使用TAB字符可能不是一个
  好的选择。)有许多的软件是用来生成随机密码的;请上网搜索。\\
  一般而言,记得一些常识就好了:不要使用某人的生日、常见的词组、桌上能
  找到的东西或其它与你自己相关的东西。当然,如``secure1''或其它一些你
  在书上或网上看到的密码也不是好密码。
\end{quote}

删除一个用户也是很容易的。只要运行\texttt{userdel}加上要删除的用户名。
删除之前要确定要删除的用户没有登陆,并且不是以要删除的用户调用
\texttt{userdel}。还有一点要记得,在删除一个用户之后,该用户的所有密码
信息都会彻底被删除。

\begin{Verbatim}[frame=single, commandchars=\\\{\}]
# \textbf{userdel jellyd}
\end{Verbatim}
这个命令会删除我们刚建立的这个讨厌的\texttt{jellyd}用户。谢天谢地!现
在这个用户已经从\path{/etc/passwd}、\path{/etc/shadow}及
\path{/etc/group}文件中删除了,但用户的主目录还没有删除。

如果你想一起删除用户的主目录,那么可以使用下面的命令:
\begin{Verbatim}[frame=single, commandchars=\\\{\}]
# \textbf{userdel -r jellyd}
\end{Verbatim}

暂时关闭某个用户会在下一节中进行介绍,因为这涉及到改变用户的密码。改变
其它的用户信息将在第
\ref{sec:systemAdministration:usersAndGroups:changingUserInformation}
节中进行介绍。

添加和删除用户组的程序十分简单。\texttt{groupadd}会在\path{/etc/group}
文件中新增一个条目,并赋予一个唯一的组ID,而\texttt{groupdel}会删除指
定的组。而是否要通过修改\path{/etc/group}文件来将某个用户添加到特定的
组中就取决于你自己了。例如,添加一个名为\texttt{cvs}的用户组:
\begin{Verbatim}[frame=single, commandchars=\\\{\}]
# \textbf{groupadd cvs}
\end{Verbatim}
而删除它则用命令:
\begin{Verbatim}[frame=single, commandchars=\\\{\}]
# \textbf{groupdel cvs}
\end{Verbatim}

\subsection{修改密码}
\label{sec:systemAdministration:usersAndGroups:changingPasswords}
\texttt{passwd}程序通过修改\path{/etc/shadow}文件来修改用户的密码。这
个文件中以加密的方式保存了系统中的所有密码。要改变用户自己的密码,只要
输入:
\begin{Verbatim}[frame=single, commandchars=\\\{\}]
\% \textbf{passwd}
Changing password for chris
Old password:
Enter the new password (minumum of 5, maximum of 127 characters)
Please use a combination of upper and lower case letters and numbers.
New password:
\end{Verbatim}

正如我们看到的,它先提示我们输入旧的密码。在输入的时候,屏幕上不会有任
何的信息,就像登陆的时候一样。之后提示让我们输入新的密码。
\texttt{passwd}会对你输入的密码做一系列的检查,如果你的密码过不了检查,
它就会给出相应的提示信息。当然,我们可以完全不管这些警告信息,之后会要
求我们再输入一次密码进行确认。

如果你使用的是\texttt{root},可以修改其它用户的密码:
\begin{Verbatim}[frame=single, commandchars=\\\{\}]
# \textbf{passwd  ted}
\end{Verbatim}
之后的过程和上面的一致,除了现在不需要输入旧的密码之外(这是使用
\texttt{root}用户的福利之一……)。

如果需要的话,我们也可以暂时关闭某个帐号,再在之后需要的时候开启它。关
闭和开启帐号都可以通过\texttt{passwd}完成。要关闭一个帐号,用
\texttt{root}执行下面的命令:
\begin{Verbatim}[frame=single, commandchars=\\\{\}]
# \textbf{passwd -l david}
\end{Verbatim}
这个命令会将david的密码改成一个加密后永远也匹配不了的密码。之后可以通
过如下命令来启用该帐号:
\begin{Verbatim}[frame=single, commandchars=\\\{\}]
# \textbf{passwd -u david}
\end{Verbatim}

现在,david的帐号又恢复正常。如果某个用户不遵守我们系统中的规定,停用
帐号的功能就显得很有用了。当然,如果你看他不爽,也可以把它停用了。


\subsection{修改用户信息}
\label{sec:systemAdministration:usersAndGroups:changingUserInformation}

用户可以修改的信息包括两个方面:使用的shell及他们的
finger\footnote{finger是一个显示用户信息的命令,这里指可以修改它能显示
的信息。}信息。Slackware中使用\texttt{chsh}(change shell,译为改变
shell)及\texttt{chfn}(change finger,改变finger信息)来修改这些值。

用户可以使用任意一个在\path{/etc/shells}文件中列出的shell。对于多数用
户,\path{/bin/bash}就足够使用了。另一些人可能对它们工作或学校中的系统
更为熟悉,并想使用那些shell。要改变默认的shell,使用\texttt{chsh}:
\begin{Verbatim}[frame=single, commandchars=\\\{\}]
\% \textbf{chsh}
Password: 
Changing the login shell for chris 
Enter the new value, or press ENTER for the default
        Login Shell [/bin/bash]:
\end{Verbatim}

在输入密码\footnote{笔者在机器上测试时貌似不需要密码……}之后,输入想要使
用的新shell的完整路径。首先请确保你输入的路径在\path{/etc/shells}(5)文
件中列出。\texttt{root}用户可以为任何人改变shell,只要在运行
\texttt{chsh}时加上要修改的用户名作为参数即可。

所谓的finger信息是一些可选的信息:如用户的全名、电话号码及房间号等。这
些信息可以通过\texttt{chfn}来改变,或在创建用户时指定。和以往一样,
\texttt{root}可以为所有用户改变该信息。

\section{用户及用户组:困难的方法}
\label{sec:systemAdministration:usersAndGroupsHardWay}
当然,因为用户及用户组的信息与控制都是通过文件进行的。所以我们也可以不
通过任何脚本或专用的程序来管理,而只通过文本编辑器手工对相应的文件进行
修改来达到管理用户及用户组的任务。在了解完它的过程后这项任务也是十分简
单的,可能比使用脚本还简单。然而,了解密码信息的实际存储方式还是很重要
的,这也可以防止你遇到需要修改这些信息,但手头上又没有相关工具的情况。

首先,我们需要将一个新的用户加入到\path{/etc/passwd}(5)、
\path{/etc/shadow}(5)及\path{/etc/group}(5)文件中。\texttt{passwd}文件
存储了系统中的用户的一些信息,但奇怪的是,其中并不包括密码信息。历史上
曾经使用该文件存放用户的密码信息,但在很早之前就已经因为安全原因停止了使
用。因为\texttt{passwd}文件必须是所有用户可读的,但同时我们并不希望将加
密的密码公开,因为一些潜在的入侵者在拥有加密的密码后,还是可以进行破解
的。取而代之,我们将加密后的密码存放在\texttt{shadow}文件,该文件只有
\texttt{root}用户可读,同时,\texttt{passwd}文件中的密码字段填写为
``x''。\texttt{group}文件则列出所有的用户组,且包含哪个用户属于哪个组
的信息。

我们可以使用\texttt{vipw}命令来安全地对\path{/etc/passwd}文件进行修改。
同样地,使用\texttt{vigr}命令安全地对\path{/etc/group}文件进行修改。而
使用\texttt{vipw -s}对\path{/etc/shadow}文件安全地修改。(``安全地''意
味着当前不会有两个用户同时对该文件进行修改。如果你是系统中唯一的管理
员,那么一般情况下也是安全的,但小心驶得万年船。)

让我们看看\path{/etc/passwd}文件的内容及如何添加一个新的用户。
\path{passwd}的典型结构如下所示:
\begin{Verbatim}[frame=single, commandchars=\\\{\}]
chris:x:1000:100:Chris Lumens,Room 2,,:/home/chris:/bin/bash
\end{Verbatim}
每个用户占用一行,每行中的每个字段都由冒号(`:')分隔。字段的含义分别为:
登陆名、加密后的密码(由于Slackware中使用shadow文件,所以每个用户的该
域都为``x'')、用户ID、组ID、可选的finger信息(由逗号分隔)、主目录及
默认shell。要创建一个新用户,只要在文件末尾添加一个新的行,在每个域中
添加相应的信息即可。

我们添加的信息要满足一定的要求,否则添加之后可能登陆不了。首先,确保密
码域为``x'',其次是用户名及用户ID都要是唯一的。再次是为用户添加一个组
ID,可以是100(Slackware中的``users''组),也可以是你自己定义的默认组
(填写的是组ID,而不是组名)。最后,为用户赋予一个有效的主目录(之后要
手工创建的)和有效的shell(记得,shell必须是\path{/etc/shells}中的一
个)。

下一个步骤,我们要在\path{/etc/shadow}中添加相应的条目。该文件中包含了
加密后的密码。一个典型的条目如下所示:
\begin{Verbatim}[frame=single, commandchars=\\\{\}]
chris:$1$w9bsw/N9$uwLr2bRER6YyBS.CAEp7R.:11055:0:99999:7:::
\end{Verbatim}
与之前一致,一个条目代表一个用户,字段是由冒号分隔的。每个字段的含义依次为:
\begin{enumerate}
\item 登陆名
\item 加密后的口令
\item 上次修改口令的时间,这个时间为从1970年1月1日算起的天数
\item 两次两次修改口令间隔最少的天数,如果这个字段的值为空,帐号永久可用
\item 两次修改口令间隔最多的天数,如果这个字段的值为空,帐号永久可用
\item 提前多少天警告用户口令将过期,如果这个字段的值为空,帐号永久可用
\item 在口令过期之后多少天禁用此用户;如果这个字段的值为空,帐号永久可用
\item 用户过期日期;此字段指定了用户作废的天数(从1970年的1月1日开始的
天数),如果这个字段的值为空,帐号永久可用
\item 保留字段,目前为空,以备将来发展之用
\end{enumerate}
可以看到,多数信息为帐号过期的信息。如果你不需要使用过期信息,那么只要
为一些字段填写某些特定的值。否则在填写之前就要经过一些思考和计算。对于
一个新用户,只要在密码字段中随意填写一些信息即可。现在不需要担心密码的
正确与否,因为我们一会儿就要修改。而密码字段中唯一不能包括的就是冒号。
将``上次修改口令的时间''也放空。之后,填上0,99999和7,就像例子中一样,
其它的字段不填。

(对于那些看到加密的密码后,想破解密码的人,请继续。如果你破解了密码,
你就会知道密码有一个加了防火墙的测试系统。\footnote{you'll know the
password to a firewalled test system.不知道是不是指密码的的加密算法。}
现在就很有用了。)

在典型的Slackware系统中,所有的普通用户都是``users''组的一员。然而,如
果你想创建一个新的组,或将新用户加到一个额外的用户组中,你就需要修改
\path{/etc/group}文件。下面是一个典型的条目:
\begin{Verbatim}[frame=single, commandchars=\\\{\}]
cvs::102:chris,logan,david,root
\end{Verbatim}
字段含义分别为,组名:组密码:组ID:组成员(逗号分隔)。创建一个新的用
户组就是在文件末尾添加新的一行,为它赋上一个唯一的组ID,并添加上属于它
的所有用户即可。新加入组的用户要先登出再登陆才能生效。

现在,我们最好用\texttt{pwck}及\texttt{grpck}命令来检查我们做的修改是
否合法。首先,用\texttt{pwck -r}和\texttt{grpck -r}:\texttt{-r}选项的
作用是不做任何的修改,使这两个命令只列出所做的修改,而不立即做出修改。
(说得很饶口,自己试试就知道了。)我们也可以根据这个命令的结果来决定是
否继续修改其它文件。不加\texttt{-r}参数执行\texttt{pwck}或
\texttt{grpck}的结果是让文件保持我们手工所做的修改。

现在,我们最好使用\texttt{passwd}命令来为用户创建合适的密码。之后再用
\texttt{mkdir}为用户创建主目录,目录要与\path{/etc/passwd}文件中填写的
内容一致。最后用\texttt{chown}将该目录的所属用户修改为新用户即可。

删除用户的过程与创建完全相反:首先删除\path{/etc/passwd}及
\path{/etc/shadow}文件中的相应条目,并从\path{/etc/group}文件中删除相
应的用户名。最后,如果愿意的话,将用户的主目录及邮件目录及crontab的条
目(如果有的话)一起删除就完成任务了。

删除用户组就更容易了:删除\path{/etc/group}文件中的相应条目即可。

\section{正确关机}
\label{sec:systemAdministration:shuttingDown}
正确关闭计算机是非常重要的。如果只是按下电源键来关机会对系统造成很大的
损害。在系统运行的时候,即使你什么都不做,系统也在使用着一些文件。还记
得吗?我们总是有一些在后台运行的进程。这些进程在管理着整个系统,并打开
了许多文件。当系统电源断开时,这些文件不能正确关闭,就可能造成一些冲突。
根据损坏的文件不同,系统也会产生不同的问题!不论是何种情况,下次系统启
动时都要做一个很长的文件系统检查来修复这些问题。

\begin{quote}
  \textbf{注意:}如果你的系统是日志文件系统,如ext3或reiserfs,那么
  系统损害时将得到一定的保护,比起不使用日志文件系统,如ext2,系统损坏
  后的启动时会花更少的时间。然而,系统的保护措施并不能成为你强制关机的
  借口。日志文件系统的目的是用来防止那些不受主观控制的事件,而不是用来
  保护因为懒惰而造成的文件损坏。
\end{quote}

不管什么情况下,当我们想重启或是关机,都要正确进行。首先,有好几种方法
来完成这个任务,所以要选择一个最有意思的(可能就是最方便的)方法。因为
关机和重启是相似的过程,所以关机的方法大多对重启也适用。

第一个方法就是通过\texttt{shutdown}(8)程序,它可能也算是最流行的方法了:
\texttt{shutdown}可以在一个特定的时间对系统重启或是关机,而且会为所有
当前登陆的用户显示一条信息,告诉他们系统正在关机。

最基本的关机命令是:
\begin{Verbatim}[frame=single, commandchars=\\\{\}]
# \textbf{shutdown -h now}
\end{Verbatim}
这个例子中,我们并不打算为其它的用户显示提示信息,他们只能看到
\texttt{shutdown}的默认信息。而``now''则表示我们预定的关机时间,而
``-h''则代表要关闭系统。对于一个多用户的系统,这并不是一个很友好的组合,
但对于自己的电脑来说可以很好地工作。用于多用户系统的一个更好的策略是提
前告诉他们要关机了:
\begin{Verbatim}[frame=single, commandchars=\\\{\}]
# \textbf{shutdown -h +60}
\end{Verbatim}

这个命令将在一小时(60分钟)内关闭系统,在一个普通的多用户系统上,这个
命令的效果就很不错了。一些重要的系统在关机之前的很长时间就做好了计划,
并且需要通过一切可能的方式来通知用户(email,通知栏,\path{/etc/motd}
等等)。

重启系统可以使用同一个命令,只要将``-h''选项改为``-r''即可:
\begin{Verbatim}[frame=single, commandchars=\\\{\}]
# \textbf{shutdown -r now}
\end{Verbatim}

其它的诸如通知信息等与\texttt{shutdown -h}中介绍的一致。
\texttt{shutdown}能在关闭或重启电脑前做很多其它的工作,请参见man手册。

另一个关机的方法是使用\texttt{halt}(8)命令及\texttt{reboot}(8)命令。就
像名字一样,\texttt{halt}会立即关闭操作系统,而\texttt{reboot}会立即重
启系统(\texttt{reboot}其实是\texttt{halt}的一个符号链接。),调用如下
例:
\begin{Verbatim}[frame=single, commandchars=\\\{\}]
# \textbf{halt}
# \textbf{reboot}
\end{Verbatim}

一个更为底层的重启或关机的方法是直接与\texttt{init}对话。所有其它的方
法只是提供一个与\texttt{init}更为方便的对话接口,但我们也可以直接使用
\texttt{telinit}(8)来告诉\texttt{init}应该怎么做(注意telinit中只有一
个`l')。使用\texttt{telinit}来告诉\texttt{init}应该进入哪个运行级别,
而进入某个运行级别时就会执行相应的脚本。这些脚本会杀死或创建一些进行以
提供进入运行级别所需要的条件。对于重启和关机而言,只是对应两个特殊的运
行级别而已。
\begin{Verbatim}[frame=single, commandchars=\\\{\}]
# \textbf{telinit 0}
\end{Verbatim}
运行级别0为关机模式。告诉\texttt{init}进入运行级别0会杀死所有当前运行
的进程、之后卸载所有的文件系统、最后关闭机器。这是一个关闭系统的很好接
受的过程。在许多笔记本及如今的桌面计算机上,这个命令的最终结果通常就是
关闭电源。

\begin{Verbatim}[frame=single, commandchars=\\\{\}]
# \textbf{telinit 6}
\end{Verbatim}
运行级别6是重启模式。所有的进行都将被杀死、文件系统将被卸载、之后机器
重新启动。这对于重启系统而言也是一个很好接受的过程。

如果你还感到困惑,那么告诉你,只要你进入运行级别0或6,不论是用
\texttt{shutdow}、\texttt{halt}还是\texttt{reboot},都会运行
\path{/etc/rc.d/rc.6}脚本。(\path{/etc/rc.d/rc.0}脚本也是
\path{/etc/rc.d/rc.6}的一个符号链接。)你也可以自定义这些文件,但切记
要仔细检查所做的任何修改。

还有最后一种重启系统的方法。其它所有的方法都要求我们以\texttt{root}用
户登陆,而这种方法即使在我们不是\texttt{root}的情况下也能重启系统,只
要你手头上有键盘就可以了。使用Ctrl+Alt+Del就会让系统立即重启(在按下这
个组合键时,系统将调用\texttt{shutdown}命令)。这个方法
在X系统下不适用,但我们可以先用Ctrl+Alt+F1(或其它功能键)切换到非X
Window终端下再使用些方法。

最后,最终控制启动和关闭的文件是\path{/etc/inittab}(5)文件。一般而言,
不需要对它进行修改,但查看该文件能让你了解到启动及关闭过程的一些细节。
一样的,请参阅man手册。


%%% Local Variables: 
%%% mode: latex
%%% TeX-master: "../SlackGuide"
%%% End: 
