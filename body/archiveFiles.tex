
\chapter{文件归档}
\label{chap:archiveFiles}
\begin{flushleft}
\rule[0mm]{\textwidth}{.1pt}
\end{flushleft}

\section{gzip}
\label{sec:archiveFiles:gzip}

\texttt{gzip}(1)是GNU的压缩程序。它的作用是压缩一个单独的文件。基本用
法如下:
\begin{Verbatim}[frame=single, commandchars=\\\{\}]
\% \textbf{gzip filename}
\end{Verbatim}

这个命令的结果是文件\texttt{filename}变成了\texttt{filename.gz},而且
文件大小一般比原文件要小。注意,使用gzip,\texttt{filename.gz}会覆盖
\texttt{filename}。这意味着\texttt{filname}文件不再存在。这个压缩算法
对于普通的文本文件效果不错,但对于jpeg图像、mp3及其它一些文件,由于它
们本身就已经压缩过了,所以效果不是很好。一般使用到的参数是用来控制压缩
大写与压缩时间的平衡的。例如,要得到最大的压缩率,可以用下面的命令:
\begin{Verbatim}[frame=single, commandchars=\\\{\}]
\% \textbf{gzip -9 filename}
\end{Verbatim}
这个命令需要花更多的时间来压缩文件,但比之前的命令得到的文件大小要更小。
而用更小的命令行参数则代表更少的压缩时间,但压缩后的大小更大。

解压缩一个\texttt{gzip}后的文件可以使用两个命令,但它们实际上是同一个
程序。\texttt{gzip}本身也可以解压缩某些扩展名的文件。这些扩展名包括:
\texttt{.gz}、\texttt{-gz}、\texttt{.z}、\texttt{-z}、\texttt{.Z}或
\texttt{-Z}。第一个方法是使用\texttt{gunzip}(1)来解压缩一个文件,如:
\begin{Verbatim}[frame=single, commandchars=\\\{\}]
\% \textbf{gunzip filename.gz}
\end{Verbatim}
这个命令是\texttt{gzip filename}的逆过程,结果就是\texttt{filename.gz}
文件被解压后的\texttt{filename}文件取代。\texttt{gunzip}其实是
\texttt{gzip}的一部分,与\texttt{gzip -d}命令等价。因此,\texttt{gzip}
一般发音为\texttt{gunzip},因为听起来更牛。

假设这样一种情况,我们想将解压后的文件内容传递给其它程序,但不想删除碑
的压缩文件,要怎么办?\texttt{zcat}命令会读取gzip文件的内容,在内存中解
压,并将解压后的内容传递到标准输出(除非你重定向了,否则指的是终端的屏
幕,重定向的内容参见第\ref{sec:shell:commandLine:redirectionAndPiping}
节)中。

\begin{Verbatim}[frame=single, commandchars=\\\{\}]
darkstar:~\$ \textbf{zcat /tmp/large_file.gz}
Wed Aug 26 10:00:38 CDT 2009
Slackware 13.0 x86 is released as stable!  Thanks to everyone who helped
make this release possible -- see the RELEASE_NOTES for the credits.
The ISOs are off to the replicator.  This time it will be a 6 CD-ROM
32-bit set and a dual-sided 32-bit/64-bit x86/x86_64 DVD.  We're taking
pre-orders now at store.slackware.com.  Please consider picking up a copy
to help support the project.  Once again, thanks to the entire Slackware
community for all the help testing and fixing things and offering
suggestions during this development cycle.
\end{Verbatim}

\section{bzip2}
\label{sec:archiveFiles:bzip2}
\textbf{bzip2}(1)是\texttt{gzip}的一个替代品,但工作的方式与
\texttt{gzip}几乎一样,只是扩展名变为\texttt{.bz2}。\texttt{bzip2}鼓吹
自己拥有更大的压缩率。但不幸的是,要达到更大的压缩率,往往花的时间更多,
因此\texttt{bzip2}一般比其它同类产品运行更长的时间。

\section{XZ/LZMA}
\label{sec:archiveFiles:bzip2}
最近一个加入Slackware的压缩工具是xz,它实现的算法是LZMA压缩算法。它比
bzip2更快且压缩得更好。事实上,由于它的速度快,质量好的特性,Slackware
已经用它取代gzip作为默认的压缩方案。不幸的是,直到写这篇文章为止,
xz还没有man手册页\footnote{译者翻译时已经有man手册页了,2012年6月},所
以想要查看它的选项,只能使用[--help]参数来查看。压缩文件用[-z]参数,解
压用[-d]参数。
\begin{Verbatim}[frame=single, commandchars=\\\{\}]
darkstar:~\$ \textbf{xz -z /tmp/large_file}
\end{Verbatim}

\section{tar}
\label{sec:archiveFiles:tar}
很好,现在我们知道了怎么使用各种压缩软件来对文件进行压缩,但它们中没有
一个能像zip一样对文件进行归档。直到现在为止。磁带归档器(Tape
Archiver),或者称为tar(1),是Slackware中最经常用到的归档程序。如同其
它归档软件, tar会生成一个单独的包含其它文件或文件夹的文件。默认情况下
它并不对生成的文件(一般称为``tarball'')进行压缩; 然而,Slackware中
包含的这个版本的tar支持很多压缩方案,包括上面提到的林林种种。

调用tar可以很简单,也可以是很复杂,一切都看你喜欢。一般的,创建一个
tarball的参数是[-cvzf]。我们将更深入的讨论。
\begin{table}[htpb]
  \centering
% begin receive orgtbl tarArguments
\begin{tabular}{r|l}
\hline\hline
参数 & 意义 \\ \hline
c & 创建一个tarball \\
x & 提取tarball的内容 \\
t & 显示tarball的内容 \\
v & 显示详细信息 \\
z & 使用gzip进行压缩 \\
j & 使用bzip2进行压缩 \\
J & 使用LZMA进行压缩 \\
p & 保留权限设置 \\
\hline\hline
\end{tabular}
% end receive orgtbl tarArguments 
  \caption{tar的各种参数}
  \label{tab:tarArguments}
\end{table}

\begin{comment}
#+ORGTBL: send tarArguments orgtbl-to-latex 
|------+-------------------|
| 参数 | 意义              |
|------+-------------------|
| c    | 创建一个tarball   |
| x    | 提取tarball的内容 |
| t    | 显示tarball的内容 |
| v    | 显示详细信息      |
| z    | 使用gzip进行压缩  |
| j    | 使用bzip2进行压缩 |
| J    | 使用LZMA进行压缩  |
| p    | 保留权限设置      |
|------+-------------------|
\end{comment}

相对于其它的程序,tar参数的顺序要求更为准确。例如,在读取或写入一个文
件时,必须要加[-f]参数,并且文件名要紧接着这个参数,考虑下面这个例子。
\begin{Verbatim}[frame=single, commandchars=\\\{\}]
darkstar:~\$ \textbf{tar -xvzf /tmp/tarball.tar.gz}
darkstar:~\$ \textbf{tar -xvfz /tmp/tarball.tar.gz}
\end{Verbatim}

上述两个例子中,第一个例子得到的结果和预期的一致,但第二个却不行,因为
这样的参数意味着让tar打开一个名为\texttt{z}的文件,而不是预期的
\texttt{/tmp/tarball.tar.gz}文件。

我们已经对参数进行了强调,现在让我们通过几个例子学习如何创建及提取
tarball。正如我们之前讲到的[-c]参数代表创建一个tarball,[-x]参数代表提
取tarball的内容。然而,如果我们想创建一个压缩的tarball,或对提取一个压
缩后的tarball中的内容,我们就必须指定正确的压缩方案。自然的,如果压根
不想对tarball进行压缩,那么可以把这些选项丢在一边。下面的命令使用gzip
压缩算法自寻一个新的tarball。一个不成文的规定是在所有的tarball文件名中
选加一个\textbf{.tar}的扩展名,再加一个所使用的压缩算法的扩展名。
\begin{Verbatim}[frame=single, commandchars=\\\{\}]
darkstar:~\$ \textbf{tar -czf /tmp/tarball.tar.gz /tmp/tarball/}
\end{Verbatim}

\section{zip}
\label{sec:archiveFiles:zip}
.zip这个扩展名不陌生吧。这些文件是把其它的文件和文件夹压缩后得到的。尽
管你在Linux中不常用到,它们在其它操作系统中仍经常使用,所以我们有时还
是会碰到。

要创建一个zip文件,你(很自然地)需要zip(1)命令。你可以用zip命令压缩一
整个文件或文件夹,只是压缩文件夹时,要用[-r]命令来使zip递归压缩文件夹
中的内容。
\begin{Verbatim}[frame=single, commandchars=\\\{\}]
darkstar:~\$ \textbf{zip -r /tmp/home.zip /home}
darkstar:~\$ \textbf{zip /tmp/large_file.zip /tmp/large_file}
\end{Verbatim}

参数的顺序是非常重要的,第一个文件名必需是要创建的zip文件(如果你不
写.zip扩展名,zip命令会自动帮你添上),剩下的参数是要加入这个zip的文件
或文件夹。

很自然地,unzip(1)命令用来解压一个zip归档文件。
\begin{Verbatim}[frame=single, commandchars=\\\{\}]
darkstar:~\$ \textbf{unzip /tmp/home.zip}
\end{Verbatim}

%%% Local Variables: 
%%% mode: latex
%%% TeX-master: "../SlackGuide"
%%% End: 
