
\chapter{帮助}
\label{chap:help}

\begin{flushleft}
\rule[0mm]{\textwidth}{.1pt}
\end{flushleft}

在配置一个程序或安装一个硬件时,我们时常需要查看某个命令的帮助。也可能
是你想对一个命令更深入地了解,或者查看它别的选项。好在我们有很多方式来
得到我们寻求的帮助。在安装Slacware时,其中的``F''系列就包含了FAQ和
HOWTO文档,你可以选择安装它。一般程序都会包含关于选项、配置文件及使用
方法的帮助。

\section{系统帮助}
\label{sec:help:systemHelp}

\subsection{man}
\label{sec:help:systemHelp:man}

\texttt{man}命令(``manual''的缩写)是Unix及Linux系统中最为传统的在线
文档。``man手册页''由特定格式的文件构成,涵盖了绝大多数命令,并与软件
本身一同发布。很自然地,执行命令\texttt{man somecommand}就显示指定命令
的man手册页,本例中,则应该显示我们虚构的命令\texttt{somecommand}的手
册页。

你可能立即会想到,man手册页的数量会急剧上升,最后连高级用户都会被弄得
晕头转向的。因此,man手册被分为几个节。这个设定已经存在很久了;久到我
们会经常看到一些命令、程序甚至是编程库函数在引用时都附上了man的节号。
例如:

你可能会看到诸如\texttt{man\textbf{(1)}}的引用。其中的数字告诉我们
``\texttt{man}''的文档在第一节(用户命令);你也可以用命令\texttt{man
  1 man}命令来为``man''指定查看手册的第一节。在一个同名的项有多个手册
页时,指定man应查看的节号是很有用的。

\begin{table}[hpbf]
  \centering
  \begin{tabular}{rl}
    \hline \hline 
    \textbf{节号} & \textbf{内容} \\ \hline 
    第~1~节 & 用户命令(只含介绍) \\
    第~2~节 & 系统调用 \\
    第~3~节 & C库调用 \\
    第~4~节 & 设备及特殊文件 \\
    第~5~节 & 文件格式及协议(即,wtmp、\texttt{/etc/passwd}及nfs等)
    \\
    第~6~节 & 游戏(只含介绍) \\
    第~7~节 & 一些约定,宏包等(即nroff,ascii等) \\
    第~8~节 & 系统管理(只含介绍) \\
    \hline \hline
    
  \end{tabular}
  \caption{man手册分节}
  \label{tab:manPageSections}
\end{table}

除了\texttt{man\textbf{(1)}},还可以使用其它命令:
\texttt{whatis\textbf{(1)}}及\texttt{apropos\textbf{(1)}}。这些命令的
目的是使我们更容易在man系统中找到有用信息。

\texttt{whatis}命令会为系统命令给出非常简洁的描述,就像放在口袋的小抄
一样。

例如:
\begin{Verbatim}[frame=single, commandchars=\\\{\}]
\$ \textbf{whatis whatis}
whatis []            (1)  - search the whatis database for complete words
\end{Verbatim}

\texttt{apropos}命令的作用是:给定一个关键词,在man手册页中搜索含有该
关键词的内容。

例如:
\begin{Verbatim}[frame=single, commandchars=\\\{\}]
\$ \textbf{apropos wav}
SDL_FreeWAV          (3)  - Frees previously opened WAV data
SDL_FreeWAV []       (3)  - Frees previously opened WAV data
SDL_LoadWAV          (3)  - Load a WAVE file
SDL_LoadWAV []       (3)  - Load a WAVE file
cdda2wav []          (1)  - a sampling utility that dumps CD audio
 data into wav sound files
cwaves []            (6)  - languid sinusoidal colors
fadeplot []          (6)  - draws a waving ribbon following a sinusoidal path
interference []      (6)  - decaying sinusoidal waves
oggdec []            (1)  - simple decoder, Ogg Vorbis file to PCM
 audio file (WAV or RAW)
pilot []             (1)  - wav - Decodes Palm Voice Memo files to
 wav files you can read on your desktop
wavelan []           (4)  - AT&T GIS WaveLAN ISA device driver
\end{Verbatim}
如果你想深入了解这些命令,请阅读它们的man手册以获取更多信息。:)

\subsection{\texttt{/usr/doc}文件夹}
\label{sec:help:systemhelp:usrDoc}

我们构建的多数软件包含有一些文档:README文件、使用说明及许可文件等。源
码中包含的任何文档都安装在系统的\texttt{/usr/doc}文件夹。每个文件都会
(一般而言是的)在\texttt{/usr/doc}中创建自己的文件夹,并将文档放在该
文件夹中,该文件夹的名字遵守如下约定:
\begin{Verbatim}[frame=single, commandchars=\\\{\}]
/usr/doc/\$program-\$version
\end{Verbatim}
其中,\texttt{\$program}表示你想查找的程序的名字,\texttt{\$version}字
段(很明显)表示在系统中安装的软件的版本号。

例如,想阅读关于命令\texttt{man\textbf{(1)}}的文档,你可能想用
\texttt{cd}命令切换到目录:
\begin{Verbatim}[frame=single, commandchars=\\\{\}]
\$ \textbf{cd /usr/doc/man-\$version}
\end{Verbatim}
如果对应的man手册页没有提供足够的信息,或者没有你想找的那些信息,那么
接下来可以考虑阅读\texttt{/usr/doc}文件夹下的相关内容。

\subsection{HOWTO及mini-HOWTO}
\label{sec:help:systemHelp:howto}

归功于开源社区的最真挚的精神,我们才有了HOWTO/mini-HOWTO。这些文件跟名
字一样,是一些如何完成某些任务的文档或指南。如果你选择了安装HOWTO文档,
那么它们会被安装在\path{/usr/doc/Linux-HOWTOs},mini-HOWTO会被安装在
\path{/usr/doc/Linux-mini-HOWTOs}。

该软件包中还含有FAQ集,FAQ是\textbf{F}requently \textbf{A}sked
\textbf{Q}uestions,中文含义为``常问问题''。这些文档是以问答的形式写成
的。如果你只是想快速地解决一个问题,那么通常选择查看FAQ是极其有用的。
如果在安装时选择安装FAQ,那么它们会被安装在
\path{/usr/doc/Linux-FAQs}目录下。

在你不知道如何处理一些东西时,这些文档通常是值得阅读的。它们涵盖了相当
大的范围,而且解决方法通常很详细,详细到不可思议。好东西是吧!

\section{在线帮助}
\label{sec:help:onlineHelp}

除了Slackware中可以提供安装的这些文档外,在网上还在大量的在线资源可供
我们学习。

\subsection{官网及论坛}
\label{sec:help:onlineHelp:website}
Slackware官网\footnote{www.slackware.com}

Slackware的官网有时比较过时,但还是有关于最新的Slackware版本的一些信息。
有一个时期官网上是有在线帮助的,但后来多了很多人来骚扰生事,导致维护这
个论坛变得很不容易,于是Patrick就把论坛关了。如果想找的话,可以在
\url{http://www.userlocal.com/phorum/}上找到之前数据的一个可搜索的归档。

在\url{http://slackware.com}上的论坛挂掉之后,一些其它的站点如雨后春笋
般崛起以提供Slackware的论坛支持。经过再三思量,Pat决定将
\url{http://www.linuxquestions.org}作为Slackware的官方论坛。

\subsection{E-mail支持}
\label{sec:help:onlineHelp:emailSupport}

据说只要购买官方的CD,就可以通过电子邮件得到开发人员的免费安装支持。话
虽如此,请记住,我们Slackware的开发人员(及绝大多数的用户)是``老派
\footnote{The Old School}''。这意味着我们更倾向于帮助那些真的有兴趣的
``自助者''。无论谁给我们发email问问题,我们都会尽我们所能帮助他们。然
而,请在发email前查阅相关的文档及网站(尤其是FAQ及下列的一些论坛)。因
此通过这些途径你可能更快得到答案,同时我们也可以少回一些email,显然,
我们也才能够更快地帮助那些需要帮助的人们。

技术支持的email是\path{support@slackware.com}\footnote{作者没有给他
  们发过邮件,不知道该邮箱的有效性。}。其它的邮箱地址及联系信息都在官
网上给出。

\subsection{Slackware的邮件列表}
\label{sec:help:onlineHelp:mailingLists}

我们有许多邮件列表,不论是摘要式的,还是正常格式的。请查阅如何订阅邮件
列表的说明。

要订阅邮件列表,请发送邮件到
\begin{verbatim}
majordomo@slackware.com
\end{verbatim}
邮件正文写上``\texttt{subscribe [列表名]}''。列表名的描述在下面(使用
下面列出的名字中的任意一个)。

邮件列表的完整归档可以在Slakcware的官网上找到:
\url{http://slackware.com/lists/archive}

\begin{description}
\item[slackware-announce] \hfill \\
 \texttt{slackware-announce} 邮件列表是关于新
  版本的声明,主要的一些更新及其它一般的信息
\item[slackware-security] \hfill \\
  \texttt{slackware-security} 邮件列表是关于安
  全问题的声明。任何直接涉及到Slackware的任何攻击或者漏洞都会立即在这
  份列表上发布。
\end{description}
这些列表还有对应的摘要版本。这意味着你可以每天得到一条大的信息而不是每
天得到一大堆的信息。由于Slackware的邮件列表并不允许用户发表信息,因此
该列表的流量很小,所以多数用户认为摘要版没什么用。但如果你需要的话,只
要订阅 \texttt{slackware-announce-digest} 或
\texttt{slackware-security-digest} 即可。

\subsection{非官方的网站及论坛}
\label{sec:help:onlineHelp:nonOfficial}

\subsubsection{网站}
\label{sec:help:onlineHelp:nonOfficial:websites}

\begin{description}
\item[Google (\url{http://www.google.com})] \hfill \\
搜索引擎的功夫大师。当你积极地、真心地想找到一个主题的每一个信息:它是
不二之选。
\item[Google:Linux(\url{http://www.google.com/linux})] \hfill \\
专门搜索Linux版块。
\item[Google:BSD(\url{http://www.google.com/bsd})] \hfill \\
专门搜索BSD的内容。Slackware是一个类Unix的通用系统,通用到经常可以在这
里找到一些与Slackware百分百相关的详细信息。很多时候,一个BSD内容的搜索
会比一般的面向大众的Linux搜索得到更多的技术细节。
\item[Google:Groups(\url{http://groups.google.com})] \hfill \\
搜索几十年来Usenet上的文章来发掘你的智慧之光。
\item[\url{http://userlocal.com}] \hfill \\
一个关于知识、好建议、第一手经验及有趣文章的宝库。通常我们会在这里听到
Slackware世界的新进展。
\end{description}

\subsubsection{网络资源}
\label{sec:help:onlineHelp:nonOfficial:webbasedResources}

\begin{description}
\item[linuxquestions.org]\footnote{\url{http://www.linuxquestions.org/questions/forumdisplay.php?forumid=14}} \hfill \\
Slackware 官方认可的论坛。
\item[LinuxISO.org Slackware论坛]\footnote{貌似和Linuxquestions.org合并
    了,新网址是\url{iso.linuxquestions.org/slackware}} \hfill \\
``一个下载Linux和寻求帮助的地方''。
\item[alt.os.linux.slackware FAQ]\footnote{貌似挂掉了。\url{http://wombat.san-francisco.ca.us/perl/fom}} \hfill \\
另一个FAQ。
\end{description}

\subsubsection{Usenet小组(NNTP)}
\label{sec:help:onlineHelp:nonOffice:usenet}

Usenet一直以来都是geek们聚集并互相帮助的地方。其中有几个致力于
Slackware的内容。但里面的人更多的是在行的人。
\begin{verbatim}
alt.os.linux.slackware
\end{verbatim}
\texttt{alt.os.linux.slackware},更为人所知的名字是aols(不要和
AOL\textregistered{}混起来!),在遇到Slackware的问题时,是获取相关的
技术帮助的最活跃的地方。就像所有的Usenet新闻组一样,一些不帮忙的参与者
(``山精''们\footnote{斯堪的那维亚神话中的,邪恶的巨怪或顽皮的侏儒})
总是糟到大家的非议。学会无视那些山精们,认出那些真心帮助别人的人,对于
使这个资源最大化利用很重要。


%%% Local Variables: 
%%% mode: latex
%%% TeX-master: "../SlackGuide"
%%% End: 
